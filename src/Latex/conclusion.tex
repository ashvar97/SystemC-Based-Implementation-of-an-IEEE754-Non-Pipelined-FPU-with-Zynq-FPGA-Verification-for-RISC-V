% Conclusion chapter for thesis
% This file will be included via % Conclusion chapter for thesis
% This file will be included via % Conclusion chapter for thesis
% This file will be included via % Conclusion chapter for thesis
% This file will be included via \input{conclusion} in main thesis.tex

\chapter{Summary}
\label{chap:summary}

This thesis has worked on a comprehensive approach to design, implementation and verification of a pipelined floating-point processor using SystemC with a path in future towards FPGA prototyping on different platform boards. Our research follows a structured development approach starting from high level design to physical hardware implementation beginning with architectural conceptualization followed by SystemC modelling and then converted to System Verilog hardware description language for synthesis and optimization to Zynq platform. The five-stage pipeline has been effective to implement IEEE 754 complaint floating operations while giving an option for adding more extensions like integer operations, branching etc. The use of high-level synthesis tools like Intel Compiler for SystemC has provided an efficient path from architectural modeling to synthesizable hardware descriptions. This approach rather than going for manual RTL coding has saved time and prevented conversion errors. The verification methodology has shown the importance of combining direct testing followed by randomized testing with thorough waveform analysis to ensure a complete test phase. While limitations do exist in the current code, particularly regarding FPGA interoperability, performance optimization and shortage of ALU operations, our research is a solid start for developing a processor with all functions in future. The identified future works including architectural extensions, implementation enhancements and methodological advancements, provide a roadmap can be used in development of the floating-point hardware. Our research provides a foundation to the field of floating-point hardware design that balance IEEE compliance.The structured methodology and insights captured in the research offer guidance for further work in specialized floating-point code with more extensions. in main thesis.tex

\chapter{Summary}
\label{chap:summary}

This thesis has worked on a comprehensive approach to design, implementation and verification of a pipelined floating-point processor using SystemC with a path in future towards FPGA prototyping on different platform boards. Our research follows a structured development approach starting from high level design to physical hardware implementation beginning with architectural conceptualization followed by SystemC modelling and then converted to System Verilog hardware description language for synthesis and optimization to Zynq platform. The five-stage pipeline has been effective to implement IEEE 754 complaint floating operations while giving an option for adding more extensions like integer operations, branching etc. The use of high-level synthesis tools like Intel Compiler for SystemC has provided an efficient path from architectural modeling to synthesizable hardware descriptions. This approach rather than going for manual RTL coding has saved time and prevented conversion errors. The verification methodology has shown the importance of combining direct testing followed by randomized testing with thorough waveform analysis to ensure a complete test phase. While limitations do exist in the current code, particularly regarding FPGA interoperability, performance optimization and shortage of ALU operations, our research is a solid start for developing a processor with all functions in future. The identified future works including architectural extensions, implementation enhancements and methodological advancements, provide a roadmap can be used in development of the floating-point hardware. Our research provides a foundation to the field of floating-point hardware design that balance IEEE compliance.The structured methodology and insights captured in the research offer guidance for further work in specialized floating-point code with more extensions. in main thesis.tex

\chapter{Summary}
\label{chap:summary}

This thesis has worked on a comprehensive approach to design, implementation and verification of a pipelined floating-point processor using SystemC with a path in future towards FPGA prototyping on different platform boards. Our research follows a structured development approach starting from high level design to physical hardware implementation beginning with architectural conceptualization followed by SystemC modelling and then converted to System Verilog hardware description language for synthesis and optimization to Zynq platform. The five-stage pipeline has been effective to implement IEEE 754 complaint floating operations while giving an option for adding more extensions like integer operations, branching etc. The use of high-level synthesis tools like Intel Compiler for SystemC has provided an efficient path from architectural modeling to synthesizable hardware descriptions. This approach rather than going for manual RTL coding has saved time and prevented conversion errors. The verification methodology has shown the importance of combining direct testing followed by randomized testing with thorough waveform analysis to ensure a complete test phase. While limitations do exist in the current code, particularly regarding FPGA interoperability, performance optimization and shortage of ALU operations, our research is a solid start for developing a processor with all functions in future. The identified future works including architectural extensions, implementation enhancements and methodological advancements, provide a roadmap can be used in development of the floating-point hardware. Our research provides a foundation to the field of floating-point hardware design that balance IEEE compliance.The structured methodology and insights captured in the research offer guidance for further work in specialized floating-point code with more extensions. in main thesis.tex

\chapter{Summary}
\label{chap:summary}

This thesis has worked on a comprehensive approach to design, implementation and verification of a pipelined floating-point processor using SystemC with a path in future towards FPGA prototyping on different platform boards. Our research follows a structured development approach starting from high level design to physical hardware implementation beginning with architectural conceptualization followed by SystemC modelling and then converted to System Verilog hardware description language for synthesis and optimization to Zynq platform. The five-stage pipeline has been effective to implement IEEE 754 complaint floating operations while giving an option for adding more extensions like integer operations, branching etc. The use of high-level synthesis tools like Intel Compiler for SystemC has provided an efficient path from architectural modeling to synthesizable hardware descriptions. This approach rather than going for manual RTL coding has saved time and prevented conversion errors. The verification methodology has shown the importance of combining direct testing followed by randomized testing with thorough waveform analysis to ensure a complete test phase. While limitations do exist in the current code, particularly regarding FPGA interoperability, performance optimization and shortage of ALU operations, our research is a solid start for developing a processor with all functions in future. The identified future works including architectural extensions, implementation enhancements and methodological advancements, provide a roadmap can be used in development of the floating-point hardware. Our research provides a foundation to the field of floating-point hardware design that balance IEEE compliance.The structured methodology and insights captured in the research offer guidance for further work in specialized floating-point code with more extensions.